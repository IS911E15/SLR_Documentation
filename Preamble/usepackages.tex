\usepackage[utf8]{inputenc}
% Gør det muligt at bruge æ, ø og å i sine .tex-filer

\usepackage[T1]{fontenc}
% Hjælper med orddeling ved æ, ø og å. Sætter fontene til at være ps-fonte, i stedet for bmp	

\usepackage{syntax}
%til BNF

\usepackage{latexsym}
% LaTeX symboler

\usepackage{xcolor,ragged2e,fix-cm}			
% Justering af elementer

\usepackage{pdfpages}
% Gør det muligt at inkludere pdf-dokumenter med kommandoen \includepdf[pages={x-y}]{fil.pdf}

\usepackage{fixltx2e}
% Retter forskellige bugs i LaTeX-kernen

\usepackage{graphicx}
% Pakke til jpeg/png billeder

\usepackage{amsmath,amssymb,stmaryrd}
% Bedre matematik og ekstra fonte

\usepackage{textcomp}
% Adgang til tekstsymboler

\usepackage{mathtools}
% Udvidelse af amsmath-pakken.

\usepackage{siunitx}
% Flot og konsistent præsentation af tal og enheder med \SI{tal}{enhed}

\usepackage{url}
% Til at sætte urler op med. Virker sammen med ref.

%\usepackage[danish]{varioref}
% Giver flere bedre mulighed for at lave krydshenvisninger på dansk

\usepackage[english]{varioref}
% Giver flere bedre mulighed for at lave krydshenvisninger på engelsk

\usepackage[round, authoryear]{natbib}
% Litteraturliste med forfatter-år og nummerede referencer

\usepackage{xr}
% Referencer til eksternt dokument med \externaldocument{<NAVN>}

\usepackage{nomencl}
% Pakke til at danne nomenklaturliste

%\usepackage[footnote,draft,danish,silent,nomargin]{fixme}
% Indsæt rettelser og lignende med \fixme{...} Med final i stedet for draft, udløses en error for hver fixme, der ikke er slettet, når rapporten bygges.

\usepackage[colorlinks, bookmarksnumbered, bookmarksdepth=4]{hyperref}
% Giver mulighed for at ens referencer bliver til klikbare hyperlinks. .. [colorlinks]{..}

%\usepackage[danish]{babel}							
% Dansk sporg, f.eks. tabel, figur og kapitel

\usepackage[english]{babel}

\usepackage{cleveref}

\usepackage[colorinlistoftodos]{todonotes}

\usepackage{pdflscape}

\usepackage{enumitem}

\usepackage{ifthen}

\usepackage{wrapfig}

\usepackage{footnote}

\usepackage[final,silent]{fixme}

%\usepackage[draft, danish,silent]{fixme}

\usepackage{float}

\usepackage{color}

\usepackage[all]{nowidow}

\usepackage{everyshi}

\usepackage{longtable}

\usepackage{lscape}

\usepackage[lined,boxed,linesnumbered]{algorithm2e}

\usepackage{listings}

\usepackage{qtree}

\usepackage{multicol}

\usepackage[framemethod=TikZ]{mdframed}

\usepackage{caption}

\usepackage{tabularx}

\usepackage{changepage}

\usepackage{rotating}

\usepackage{lastpage}

\usepackage{intcalc}

\usepackage{nth}

\usepackage{newfloat}

\usepackage{tikz}

\usepackage{wrapfig}
